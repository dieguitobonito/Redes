% Created 2023-02-24 Fri 13:43
% Intended LaTeX compiler: pdflatex
\documentclass[11pt]{article}
\usepackage[utf8]{inputenc}
\usepackage[T1]{fontenc}
\usepackage{graphicx}
\usepackage{longtable}
\usepackage{wrapfig}
\usepackage{rotating}
\usepackage[normalem]{ulem}
\usepackage{amsmath}
\usepackage{amssymb}
\usepackage{capt-of}
\usepackage{hyperref}
\begin{center}
	\includegraphics[scale=0.25]{/home/dieguito/Pictures/logo_udeg_negro.png}\\
	\vspace{1cm}
{\Large 	\textbf{Centro Universitario de Ciencias Exactas e Ingenierías}\\}

\vspace{0.5cm}
	
{\large 	\textit{Departamento de ciencias computacionales}\\}
	
	\vspace{1cm}
	
	\textbf{Administración de redes}\\
	
	\vspace{0.5cm}
	
	\textbf{Reporte semanal}\\
	
	\vspace{0.5cm}
	
	Semana \textbf{7}
	
	\vspace{0.5cm}
	
	\underline{Packet tracer}\\
	
	\vspace{1cm}
	
	Prof: Ing. Luis Ignacio Sánchez Salazar\\
	
	Alumno: Diego Martín Domínguez Hernández\\
	
	Carrera: Ingeniería Informática \\
	
	Materia: i5907 (Administración de Redes)\\
	
	NRC: 42241\\
	
	Sección: D04\\

	Calendario: 2023A\\
	
\end{center}
\newpage

\author{Diego Domínguez}
\date{\today}
\title{}
\hypersetup{
 pdfauthor={Diego Domínguez},
 pdftitle={},
 pdfkeywords={},
 pdfsubject={},
 pdfcreator={Emacs 28.2 (Org mode 9.6)}, 
 pdflang={English}}
\begin{document}

La dirección IP es el \emph{identificador} de un
dispositivo dentro de una red, este debe ser único
y debe seguir el protocolo de red IP.

Existe el famoso protocolo de direccionamiento
IPv4, sin embargo, las direcciones otorgadas por
eset protocolo han sido agotadas, por lo tanto, el
uso de un protocolo, llamado IPv6, es el nuevo
estándar.

Lo que tienen en común ambos protocolos es que
ambos:
\begin{itemize}
\item Se encargan de otorgar direcciones IP únicas a
cada dispositivo de la red.
\item Ambos pueden asignar direcciones estáticas o
dinámicas.
\end{itemize}

Una dirección IP, es un número de 32 bits.
Para hacerlo más legible a los humanos, es
convención separar esos 32 números en grupos de 8:

Ejemplo: $$11000000101010000000000000000001$$

Se separaría en: $$11000000.10101000.00000000.00000001$$

Así, del número separado en 8 dígitos podemos
representarlo a sistema decimal, quedando: $$192.168.0.1$$


En IPv4, cada uno de los campos representa un
número en el rango de 1 byte (8 bits): 0 - 255.

Una de las características de las direcciones IP
es que contienten dos componentes:
\begin{itemize}
\item El prefijo de red (bits que identifican a la
red).
\item El identificador del dispositivo (bits que
identifican al dispositivo que está en esa red).
\end{itemize}

Se intuye que todos los dispositivos conectados a
la misma red tendrán el \textbf{mismo prefijo}

Si una red:
\begin{itemize}
\item Solo utiliza el primer byte para identificarse,
es una red de \textbf{clase A}. Si el primer octeto
está entre \emph{1 y 127} es de esta clase.
\item Utiliza dos bytes para identificarse, es \textbf{clase
B}. Si el primer octeto está entre \emph{128 y 191} es de
esta clase.
\item Utiliza 3 bytes, \textbf{clase C}. El primer octeto
está entre \emph{192 y 223}.
\end{itemize}

Como las dos partes de una dirección IP son de
longitud variable y complementan uno del otro, es
necesario implementar otro mecanismo para
delimitar los componentes, se utiliza la \textbf{máscara
de subred}

Al igual que una dirección IP, una máscara es un
número de 32 bits, pero este \textbf{no} es una dirección
IP:
\begin{itemize}
\item No se representa igual
\item No tiene la misma función
\end{itemize}
La máscara es un número que siempre está asociado
a la IP, y nos indica a partir de qué números
forman el prefijo y qué números forman el identificador.

Ejemplo, se tiene la siguiente IP clase C:
$$192.168.0.10$$
Donde \(10\) es el identificador del dispositivo.

Una máscara deberá representar la dirección IP de
la siguiente manera:
$$255.255.255.0$$

Donde los primeros 3 bytes de la máscara estarán a
\textbf{1} (representando así la clase C).
Por lo tanto, se intuye que si se necesita una
máscara de una red clase B, la máscara deberá
establecer a \textbf{1} los 2 primeros bytes. Misma idea
con las redes de clase A.

Otra forma de hacer notación de la máscara, es con
la notación CIDR:

Sea la IP:
$$192.168.0.10$$

Y su máscara:
$$255.255.255.0$$

Su notación CIDR sería:
$$192.168.0.10 / 24$$

Donde:
\begin{itemize}
\item La parte izquiera es la dirección IP.
\item La parte derecha representan los bits de la
máscara que están a 1.
\end{itemize}

Intuyendo:
\begin{itemize}
\item \(172.20.10.54 / 16\) es una IP clase B.
\item \(10.20.30.40 / 8\) es una IP clase A.
\end{itemize}
\end{document}